% In this file you should put the actual content of the blueprint.
% It will be used both by the web and the print version.
% It should *not* include the \begin{document}
%
% If you want to split the blueprint content into several files then
% the current file can be a simple sequence of \input. Otherwise It
% can start with a \section or \chapter for instance.


\section{Definitions}

\begin{definition}[Sequence]
  \label{def:sequence}
  A sequence, denoted $a$ or $\{a_n\}$, is a function $a : ℕ → \mathbb{R}$.
\end{definition}

\begin{definition}[Convergence]
  \label{def:convergence}
  \lean{convergesTo, converges}
  \leanok

  A sequence $\{a_n\}$ converges to $L ∈ \mathbb{R}$ if for all $\varepsilon > 0$ there exists
  $N ∈ ℕ$ such that for all $n ≥ N$, $|a_n - L| < \varepsilon$.
  We say $\{a_n\}$ converges if there exists $L ∈ \mathbb{R}$ such that $\{a_n\}$ converges to $L$.

\end{definition}

\begin{definition}[Bounded Above]
  \label{def:bounded_above}
  \lean{upper_bound, bounded_above}
  \leanok
  A set $S \subseteq \mathbb{R}$ is bounded above if there exists an $M \in \mathbb{R}$ such that for all $s \in S$, $s \le M$.
  We say that $M$ is an upper bound for $S$.
\end{definition}

\begin{definition}[supremum]
  \label{def:supremum}
  \uses{def:bounded_above}
  \lean{sup}
  \leanok
  Let $S \subseteq \mathbb{R}$ be non-empty. We say that $y \in \mathbb{R}$ is the supremum of $S$ if \\ \\
  (i) $y$ is an upper bound for $S$ \\
  (ii) for any upper bound $z$ of $S$, $y \le z$
\end{definition}

\begin{definition}[continuous]
  \label{def:continuous}
  \lean{continuous}
  \leanok
  A function $f : \mathbb{R} → \mathbb{R}$ is continuous if for all $x \in \mathbb{R}$ and for all $\varepsilon > 0$, there exists a $\delta > 0$ such that for all $y ∈ \mathbb{R}$ with $|x - y| < \delta$, $|f(x) - f(y)| < \varepsilon$.
\end{definition}


\section{Theorems}


\begin{theorem}[Limit Laws]
  \label{thm:limit_laws}
  \uses{def:convergence, def:sequence}
  \lean{convergesTo_scalar_mul, convergesTo_add}
  \leanok

  Let $C \in \mathbb{R}$. Suppose $\{a_n\}$ converges to $L$ and $\{b_n\}$ converges to $K$. Then \\ \\
  (i) $\{C a_n\}$ converges to $C L$ \\
  (ii) $\{a_n + b_n\}$ converges to $L + K$ 

\end{theorem}

\begin{proof}
  \leanok
  Trivial.
\end{proof}


\begin{lemma}
\label{lem:converges_nonneg}
  \uses{def:convergence, thm:limit_laws}
  \lean{convergesTo_nonneg}
  \leanok

  Suppose that there exists an $N \in ℕ$ such that for all $n \ge N$, $a_n ≥ 0$.
  Then $\lim_{n \to \infty} a_n ≥ 0$.

\end{lemma}

\begin{theorem}[Order Limit Theorem]
  \label{thm:order_limit}
  \uses{lem:converges_nonneg}
  \lean{le_convergesTo_of_le}
  \leanok

  Let $\{a_n\}$ and $\{b_n\}$ be sequences. 
  Suppose that there exists an $N ∈ ℕ$ such that for all $n ≥ N$, $a_n ≤ b_n$.
  Then $\lim_{n \to \infty} a_n \le \lim_{n \to \infty} b_n$.
\end{theorem}

\begin{theorem}[Completeness Axiom]
  \label{thm:exists_sup_of_bounded_above}
  \uses{def:supremum, def:bounded_above}
  \lean{exists_sup_of_bounded_above}
  \leanok
  Let $S \subseteq \mathbb{R}$ be non-empty and bounded above.
  Then there exists a $y \in \mathbb{R}$ such that $y$ is the supremum of $S$.

\end{theorem}

\begin{theorem}[Intermediate Value Theorem]
  \label{thm:intermediate_value}
  \uses{thm:exists_sup_of_bounded_above, def:continuous, thm:order_limit}
  \lean{intermediate_value}
  \leanok

  Let $f : \mathbb{R} → \mathbb{R}$ be continuous and let $a < b ∈ \mathbb{R}$.
  Suppose that $f(a) < f(b)$ and let $y \in [f(a), f(b)]$.
  Then there exists a $c ∈ [a, b]$ such that $f(c) = y$.
\end{theorem}

\begin{proof}
  \leanok
  Sketch: \\
  Let $K = \{x \in [a, b] | f(x) \le y\}$.
  Then $K$ is non-empty and bounded above by $b$, so it has a supremum $c \in \mathbb{R}$.
  If $f(c) < y$, then by continuity of $f$, there exists a $\delta > 0$ such that for all $x ∈ (c - \delta, c + \delta)$, $f(x) < y$.
  Choose $x = c + \delta/2$.Then $x ∈ K$ and $x > c$, so $c$ is not an upper bound for $K$.
  If $f(c) > y$, then by continuity of $f$, there exists a $\delta > 0$ such that for all $x ∈ (c - \delta, c + \delta)$, $f(x) > y$.
  Choose $x = c - \delta/2$. Then $x$ is an upper bound for $K$ and $x < c$, so $c$ is not the supremum of $K$.
  Thus, $f(c) = y$.

\end{proof}

\section{Modular Forms}

\begin{definition} [Modular Form]
  \label{def:ModularForm}
  \lean{ModularFormDefs.Regular.ModularForm}
  \leanok
  In lean, A modular form of weight $k \in \mathbb{N}$ is a function $f : \mathbb{C} → \mathbb{C}$ such that : \\ \\
  (1) $f$ is holomorphic on $\mathbb{H}$ \\
  (2) For all $z \in \mathbb{H}, f(z) = f(z + 1)$ \\
  (3) For all $z \in \mathbb{H}, f(z) = z^{-k} f(-1/z)$ \\
  (4) $f$ is bounded as Re$(z) \to \infty$ \\
\end{definition}

\begin{definition} [Integer Modular Form]
  \label{def:IntegerModularForm}
  \uses{def:ModularForm}
  \lean{ModularFormDefs.Integer.IntegerModularForm}
  \leanok
  An integer modular form of weight $k \in \mathbb{N}$ is a sequence $a : \mathbb{N} → \mathbb{Z}$
  such that $\sum_{n=0}^{\infty} a(n) q^n$ is a modular form of weight $k$, where $q = e ^ {2 \pi i z}.$
\end{definition}